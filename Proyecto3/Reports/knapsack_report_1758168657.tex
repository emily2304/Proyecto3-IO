\documentclass{article}
\usepackage[utf8]{inputenc}
\usepackage{amsmath}
\usepackage{xcolor}
\usepackage{geometry}
\geometry{margin=1in}
\title{Problema de la Mochila - Solución}
\author{Grupo de Trabajo}
\date{\today}
\begin{document}
\maketitle

\section*{Problema de la Mochila}
El problema de la mochila consiste en seleccionar un conjunto de objetos de tal manera que se maximice el valor total sin exceder la capacidad máxima.\\

\textbf{Tipo de problema:} Unbounded Knapsack\\
\textbf{Capacidad máxima:} 15\\
\textbf{Número de objetos:} 2\\

\section*{Datos del Problema}
\begin{tabular}{|c|c|c|c|}
\hline
Objeto & Costo & Valor & Cantidad \\
\hline
A & 3,00 & 7,00 & $\infty$ \\
B & 4,00 & 9,00 & $\infty$ \\
\hline
\end{tabular}

\section*{Solución Óptima}
\textbf{Valor máximo obtenido:} 35\\
\textbf{Capacidad utilizada:} 15\\
\end{document}
