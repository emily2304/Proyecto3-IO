\documentclass{article}
\usepackage[utf8]{inputenc}
\usepackage{amsmath}
\usepackage{xcolor}
\usepackage{geometry}
\geometry{margin=1in}
\title{Problema de la Mochila - Solución}
\author{Grupo de Trabajo}
\date{\today}
\begin{document}
\maketitle

\section*{Problema de la Mochila}
El problema de la mochila consiste en seleccionar un conjunto de objetos de tal manera que se maximice el valor total sin exceder la capacidad máxima.\\

\textbf{Tipo de problema:} 0/1 Knapsack\\
\textbf{Capacidad máxima:} 10\\
\textbf{Número de objetos:} 3\\

\section*{Datos del Problema}
\begin{tabular}{|c|c|c|c|}
\hline
Objeto & Costo & Valor & Cantidad \\
\hline
A & 4,00 & 12,00 & 1 \\
B & 6,00 & 18,00 & 1 \\
C & 3,00 & 8,00 & 1 \\
\hline
\end{tabular}

\section*{Solución Óptima}
\textbf{Valor máximo obtenido:} 30\\
\textbf{Capacidad utilizada:} 10\\
\end{document}
