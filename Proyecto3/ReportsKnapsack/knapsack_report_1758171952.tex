\documentclass{article}
\usepackage[utf8]{inputenc}
\usepackage{amsmath}
\usepackage{xcolor}
\usepackage{colortbl}
\usepackage{geometry}
\usepackage{multirow}
\usepackage{graphicx}
\geometry{margin=1in}
\definecolor{verde}{RGB}{0, 128, 0}
\definecolor{rojo}{RGB}{255, 0, 0}
\title{Proyecto 1: Rutas Optimas (Algoritmo de Floyd)}
\author{Emily Sanchez \\ Viviana Vargas \\[1cm] Curso: Investigación de Operaciones \\ II Semestre 2025}
\date{\today}

\begin{document}
\maketitle

\thispagestyle{empty}
\newpage
\setcounter{page}{1}

\textbf{Tipo de problema:} 0/1 Knapsack\\
\textbf{Capacidad máxima:} 8\\
\textbf{Número de objetos:} 2\\

\clearpage
\section*{Datos del Problema}
\begin{tabular}{|c|c|c|c|}
\hline
Objeto & Costo & Valor & Cantidad \\
\hline
A & 3,00 & 2,00 & 1 \\
B & 4,00 & 5,00 & 1 \\
\hline
\end{tabular}

\section*{Tabla de Programación Dinámica}
\begin{center}
\scriptsize
\begin{tabular}{|c|c|c|c|}
\hline
Capacidad/Objetos & Ninguno & A & B \\ \hline
0 & \cellcolor{rojo}\textcolor{white}{0} & \cellcolor{rojo}\textcolor{white}{0} & \cellcolor{rojo}\textcolor{white}{0} \\ \hline
1 & \cellcolor{rojo}\textcolor{white}{0} & \cellcolor{rojo}\textcolor{white}{0} & \cellcolor{rojo}\textcolor{white}{0} \\ \hline
2 & \cellcolor{rojo}\textcolor{white}{0} & \cellcolor{rojo}\textcolor{white}{0} & \cellcolor{rojo}\textcolor{white}{0} \\ \hline
3 & \cellcolor{rojo}\textcolor{white}{0} & \cellcolor{verde}\textcolor{white}{2} & \cellcolor{rojo}\textcolor{white}{2} \\ \hline
4 & \cellcolor{rojo}\textcolor{white}{0} & \cellcolor{verde}\textcolor{white}{2} & \cellcolor{verde}\textcolor{white}{5} \\ \hline
5 & \cellcolor{rojo}\textcolor{white}{0} & \cellcolor{verde}\textcolor{white}{2} & \cellcolor{verde}\textcolor{white}{5} \\ \hline
6 & \cellcolor{rojo}\textcolor{white}{0} & \cellcolor{verde}\textcolor{white}{2} & \cellcolor{verde}\textcolor{white}{5} \\ \hline
7 & \cellcolor{rojo}\textcolor{white}{0} & \cellcolor{verde}\textcolor{white}{2} & \cellcolor{verde}\textcolor{white}{7} \\ \hline
8 & \cellcolor{rojo}\textcolor{white}{0} & \cellcolor{verde}\textcolor{white}{2} & \cellcolor{verde}\textcolor{white}{7} \\ \hline
\end{tabular}
\end{center}
\normalsize

\section*{Solución Óptima}
\textbf{Valor máximo obtenido:} 7\\
\textbf{Objetos seleccionados:} B, A\\
\textbf{Capacidad utilizada:} 7\\
\end{document}
