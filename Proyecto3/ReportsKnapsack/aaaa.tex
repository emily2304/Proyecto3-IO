\documentclass{article}
\usepackage[utf8]{inputenc}
\usepackage{amsmath}
\usepackage{xcolor}
\usepackage{colortbl}
\usepackage{geometry}
\usepackage{multirow}
\usepackage{graphicx}
\geometry{margin=1in}
\definecolor{verde}{RGB}{0, 128, 0}
\definecolor{rojo}{RGB}{255, 0, 0}
\title{Proyecto 2: El Problema de la Mochila}
\author{Emily Sanchez \\ Viviana Vargas \\[1cm] Curso: Investigación de Operaciones \\ II Semestre 2025}
\date{\today}

\begin{document}
\maketitle

\thispagestyle{empty}
\newpage
\setcounter{page}{1}

\section{Problema de la Mochila (Knapsack Problem)}

El \textbf{problema de la mochila} es un clasico de la \textit{optimizacion combinatoria}. Se dispone de una \textbf{mochila} con una \textbf{capacidad maxima} $W$ y un conjunto de $n$ objetos. Cada objeto $i$ tiene un \textbf{peso} $w_i$ y un \textbf{valor} $v_i$. El objetivo es seleccionar los objetos de manera que:
\begin{itemize}
  \item La suma total de los pesos no exceda la capacidad $W$.
  \item Se maximice el valor total de los objetos elegidos.
\end{itemize}

\subsection{Variantes principales}
\begin{description}
  \item[0/1 Knapsack] Cada objeto puede elegirse una sola vez o no elegirse: decision binaria.
  \item[Bounded Knapsack] Cada objeto puede seleccionarse un numero limitado de veces.
  \item[Unbounded Knapsack] Se permite una cantidad ilimitada de cada objeto.
\end{description}

\subsection{Solucion}
\paragraph{Bounded Knapsack} Se resuelve con programación dinámica considerando las cantidades límite de cada objeto:
\[
dp[i][w] = \max_{k=0}^{\min(q_i, \lfloor w/w_i \rfloor)} \left( dp[i-1][w - k \cdot w_i] + k \cdot v_i \right)
\]
donde $q_i$ es la cantidad máxima del objeto $i$.

\thispagestyle{empty}
\newpage
\textbf{Tipo de problema:} Bounded Knapsack\\
\textbf{Capacidad máxima:} 10\\
\textbf{Número de objetos:} 3\\

\section*{Formulación Matemática}
\textbf{Función objetivo:}\\
Maximizar $Z = 11 x_{A} + 7 x_{B} + 12 x_{C}$\\

\textbf{Restricción:}\\
$4 x_{A} + 3 x_{B} + 5 x_{C} \leq 10$\\

\textbf{Restricciones de variables:}\\
$x_i \in \mathbb{Z}^+ \quad \forall i \in \{A, B, C\}$\\
\vspace{0.5cm}

\section*{Datos del Problema}
\begin{tabular}{|c|c|c|c|}
\hline
Objeto & Costo & Valor & Cantidad \\
\hline
A & 4,00 & 11,00 & 1 \\
B & 3,00 & 7,00 & 1 \\
C & 5,00 & 12,00 & 1 \\
\hline
\end{tabular}

\section*{Tabla de Programación Dinámica Detallada}
\begin{center}
\scriptsize
\begin{tabular}{|c|c|c|c|c|}
\hline
Capacidad & Inicial & A & B & C \\ \hline
0 & 0 & \cellcolor{rojo}\textcolor{white}{0} & \cellcolor{rojo}\textcolor{white}{0} & \cellcolor{rojo}\textcolor{white}{0} \\ \hline
1 & 0 & \cellcolor{rojo}\textcolor{white}{0} & \cellcolor{rojo}\textcolor{white}{0} & \cellcolor{rojo}\textcolor{white}{0} \\ \hline
2 & 0 & \cellcolor{rojo}\textcolor{white}{0} & \cellcolor{rojo}\textcolor{white}{0} & \cellcolor{rojo}\textcolor{white}{0} \\ \hline
3 & 0 & \cellcolor{rojo}\textcolor{white}{0} & \cellcolor{verde}\textcolor{white}{7(1)} & \cellcolor{rojo}\textcolor{white}{7} \\ \hline
4 & 0 & \cellcolor{verde}\textcolor{white}{11(1)} & \cellcolor{rojo}\textcolor{white}{11} & \cellcolor{rojo}\textcolor{white}{11} \\ \hline
5 & 0 & \cellcolor{verde}\textcolor{white}{11(1)} & \cellcolor{rojo}\textcolor{white}{11} & \cellcolor{verde}\textcolor{white}{12(1)} \\ \hline
6 & 0 & \cellcolor{verde}\textcolor{white}{11(1)} & \cellcolor{rojo}\textcolor{white}{11} & \cellcolor{verde}\textcolor{white}{12(1)} \\ \hline
7 & 0 & \cellcolor{verde}\textcolor{white}{11(1)} & \cellcolor{verde}\textcolor{white}{18(1)} & \cellcolor{rojo}\textcolor{white}{18} \\ \hline
8 & 0 & \cellcolor{verde}\textcolor{white}{11(1)} & \cellcolor{verde}\textcolor{white}{18(1)} & \cellcolor{verde}\textcolor{white}{19(1)} \\ \hline
9 & 0 & \cellcolor{verde}\textcolor{white}{11(1)} & \cellcolor{verde}\textcolor{white}{18(1)} & \cellcolor{verde}\textcolor{white}{23(1)} \\ \hline
10 & 0 & \cellcolor{verde}\textcolor{white}{11(1)} & \cellcolor{verde}\textcolor{white}{18(1)} & \cellcolor{verde}\textcolor{white}{23(1)} \\ \hline
\end{tabular}
\end{center}
\normalsize

\section*{Solución Óptima}
\textbf{Valor máximo obtenido:} 23\\
\textbf{Objetos seleccionados:} C:1, A:1\\
\textbf{Capacidad utilizada:} 9\\
\textbf{Capacidad restante:} 1\\
\end{document}
